% Chapter Template

\chapter{Introduction} % Main chapter title

\section{Iterative Filtering}\label{sec:IFBack}
Iterative filtering (IF) is a class of algorithms designed to aggregate data from several sources of unknown variances. These algorithms were initially investigated within the field of physics to improve the reliability of results. They have since been generalised and are increasingly applied to online voting systems \cite{laureti2006information}. This is especially important for preventing vote manipulation in cases where there are financial repercussion, for example eBay or Amazon product reviews. Due to their high resistance to cheating, IF algorithms are able to produce accurate and reliable results even in the case of a high noise to signal ratio \cite{juselius2014testing}, which is quite important in the case of large datasets.

Early IF algorithms typically made use of a reciprocal variance weighting of the form \cite{laureti2006information, de2007iterative, de2010iterative}:

\begin{equation}
    \tilde{\mu} = \sum\limits_i\frac{r_i\sigma_i^{-\alpha}}{\sum\limits_j \sigma_j^{-\alpha}} \label{eq:negRecip}
\end{equation}

Where $\tilde{\mu}$ is the estimated ``true'' value, $alpha$ is a positive factor that can be increased in order to reduce the influence of outliers \cite{ifNotes4121}, and $r_i$ and $\sigma_i$ are the rating and standard deviation of agent $i$, respectively. This specification does not fit most use cases of IF, as the raison d'\^etre is to calculate a minimally variant aggregate value when variance of each agent's votes is unknown, so, in this case, variance would be estimated as in equation \ref{eq:varCalc}.

\begin{equation}
\sigma_i^2 = \frac{1}{|j:i\rightarrow j| - 1}\sum\limits_{j:i\rightarrow j}(r_{i\rightarrow j} - \tilde{\mu}_j)^2 \label{eq:varCalc}
\end{equation}

Where $j:i\rightarrow j$ refers to every election $j$ where actor $i$ casts a vote and $r_{i\rightarrow j}$ refers to the vote cast by actor $i$ in election $j$. 

The basic premise of IF is to manipulate a simple average through the combined use of a weighting equation such as \ref{eq:varCalc} and a subsequently weighted average equation as seen in equation \ref{eq:negRecip}. This process is then repeated until the weighted mean converges on a static value \cite{laureti2006information}.

Some of the advantages of IF are:
\begin{description}
    \item[Accuracy] Data aggregated using the reciprocal variance IF method outlined in equations \ref{eq:negRecip} and \ref{eq:varCalc} can be shown to have a lower variance than any individual source and approaches the Cramer-Rao lower bound of variance \cite{ifNotes4121}.
    \item[Bias] Estimation errors manifest in several forms, systematic biases are particularly harmful as they lead to the consistent under of over estimation of analyst recommendations. IF can account for these errors and potentially make use of them to improve estimates \cite{ignjatovic2009computing}.
    \item[Cheating] The incentive for a security to be incorrectly priced in a malpractice that is occasionaly used by corporations to increase profits. Recently exposed to be occurring at DeutscheBank \cite{sillyDeutsche}, this fraudulent behaviour is seen in many areas of the finance industry, especially when prop trading\footnote{Prop trading refers to proprietary asset trading where a firm invests its own capital to make a profit} falls under the same corporate structure as research houses. This IF system aims to identify corporations with unusually hi variances and flag them for possible fraudulent activity\footnote{Such variances may also be explained by poor job performance.}.
    \item[Spam] Research in financial and economic areas is fraught with issues regarding misbehaving data and low signal to noise ratios. While many econometric techniques were created just to address this issue \cite{juselius2014testing}, IF can proficiently identify and disregard spam.
\end{description}

\clearpage
\section{Financial Data}
Finance is a field that has proven to be particularly resistant to study due to the extremely large volume of data available and a very high noise to signal ratio\cite{juselius2014testing}. As identified in section \ref{sec:IFBack}, using IF to aggregate financial data could potentially identify patterns of analyst skill that may have otherwise been obscured by bias or white noise.

The specific form of the data of interest is portrayed as  5-point recommendations regarding the predicted price movements of a stock. These recommendations are typically articulated as \cite{ibesMan}:
\begin{description}
    \item[5] Strong Buy
    \item[4] Buy
    \item[3] Hold
    \item[2] Sell
    \item[1] Strong Sell
\end{description}
Thus, a weighted average should be an accurate predictor of the change in stock price in the short-medium term.

The concept of ``alpha'' is mentioned quite often in the financial world and refers to an ability to reliably perform differently to a simple market aggregate \cite{blanchett2013alpha, holmes2009seeking}. This alpha could be negative, (i.e. consistently underperforming the market), or positive, (i.e. consistently outperforming). In this case ``the market'' typically refers to some market index, such as the Dow Jones in the U.S. or the ASX100 in Australia.

Alpha can be attributed to either an individual, a research firm, or any other existing sub-structure \cite{hung2009defining}. For example, the hypothetical firm ``SchmitiBank'' may provide a highly accurate rating for the Consumer Discretionary company ``Pear Inc.'' This may be a result one exceptional analyst, a highly effective Consumer Discretionary analytical team, or exceptional utilisation of all resources company wide. If either of the latter two are accredited for the rating then ``SchmitiBank'' is likely to issue accurate recommendations for all Consumer Discretionary companies or all companies, respectively. As a result of this any algorithm attempting to determine the origin of alpha within a firm would have to observe the firm's performance within a single stock, as a firm in general, and within each of the following GICS sectors:
\begin{itemize}
    \item Energy
    \item Materials
    \item Industrials
    \item Consumer Discretionary
    \item Consumer Staples
    \item Health Care
    \item Financials
    \item Information Technology
    \item Telecommunication Services
    \item Utilities
    \item Real Estate
\end{itemize}
These classifications are very useful within financial research as most companies are succinctly defined by one category. They represent the arrangement of teams in most analytical firms and, thusm are ideal for providing a proxy sub-structure to which alpha can be assigned \cite{robert2003strategic}. 